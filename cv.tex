\documentclass[11pt,a4paper,roman,russian]{moderncv} % possible options include font size ('10pt', '11pt' and '12pt'), paper size ('a4paper', 'letterpaper', 'a5paper', 'legalpaper', 'executivepaper' and 'landscape') and font family ('sans' and 'roman')
\moderncvstyle{modernclassic} % style options are 'casual' (default), 'classic', 'oldstyle' and 'banking'
\moderncvcolor{blue}                               % color options 'blue' (default), 'orange', 'green', 'red', 'purple', 'grey' and 'black'
%\nopagenumbers{}                                  % uncomment to suppress automatic page numbering for CVs longer than one page
\usemoderncventry

\usepackage[scale=0.8,a4paper]{geometry}
\usepackage{xltxtra}
\usepackage{xunicode}
\setmainfont[Mapping=tex-text]{Times New Roman}
\setsansfont[Mapping=tex-text]{Calibri}
\newfontfamily{\cyrillicfonttt}{Courier New}
\usepackage{polyglossia}
\setmainlanguage{russian}
\setotherlanguage[variant=us]{english}

%\newcommand\cventryitem{\item[\textbullet]\hspace*{1em}}
\usepackage{enumitem}
\setlist{nolistsep} % remove extra space between items
\setlist[itemize,1]{labelindent=6pt, leftmargin=*}
\newcommand{\cventryitem}{\item[\textbullet]}

\newlength{\cventryvspace}
\setlength{\cventryvspace}{0.75em}

%----------------------------------------------------------------------------------
%            personal data
%----------------------------------------------------------------------------------
\firstname{Владимир}
\familyname{Панченко}
\title{Software Developer}
\vacancy{Разработчик C\# (web, back-end)}
\bannerquote{“Any fool can write code that a computer can understand. Good programmers write code that humans can understand.”~---~Martin Fowler}
%\address{г. Москва}
\mobile{+7 (926) 410-55-70}
\email{dropsonic.pan@gmail.com}
\skype{volodymyrthegreat}
\github{dropsonic}

\photo[100pt][0pt]{picture}

\begin{document}
\makecvtitle

\section{Ключевые навыки}
\cvitem{}{Проектирование и разработка сложных cloud-based информационных систем}
\cvitem{}{Организация процесса разработки ПО с использованием Agile-методологий}
\cvitem{}{Управление командами разработки, включая собеседование разработчиков и аналитиков}
\cvitem{}{Управление продуктом (product management), сбор требований и работа с клиентами}

\section{Технологии}
\cvitem{}{C\#, .NET Framework / .NET Core, SQL (SQL Server, MySQL), Docker, RabbitMQ, ASP.NET MVC / WebAPI, OWIN, SignalR, Seq}
\cvitem{}{Entity Framework, Autofac, Serilog, Automapper, Visual Studio SDK, xUnit, Moq, FluentAssertions, RestSharp, EasyNetQ, MassTransit, Azure IoT SDK, Selenium}
\cvitem{}{Microsoft Azure, Amazon AWS}
\cvitem{}{Git, Mercurial, Visual Studio, JetBrains ReSharper / dotTrace / dotMemory, Azure Pipelines, PowerShell, Bash, \LaTeX, Atlassian Jira, BitBucket, Bamboo, Confluence}
\cvitem{}{Хороший уровень владения ООП, SOLID, шаблонами проектирования, рефакторингом, DDD, UML. Владение Agile-методологиями (Scrum, Kanban, XP), TDD, написание юнит-тестов}

\section{Опыт работы}
\cventry[\cventryvspace]{07.2017~--~н.в.}{}{Acumatica}{Team Lead}{Reporting~\&~Core}
{Управление двумя командами разработки в следующих продуктах:
\begin{itemize}[label=\textbullet]
  \item \href{https://www.acumatica.com/}{Облачная ERP-система (SaaS) для малого и среднего бизнеса}, основные области: reporting и платформенный функционал (4 разработчика + аналитик)
  \item Acuminator: VSIX-расширение для Visual Studio для статического анализа кода и облегчения разработки (2 разработчика + технический писатель). \href{https://github.com/Acumatica/Acuminator}{Доступен на GitHub \faGithub}
\end{itemize}
Основные обязанности:
\begin{itemize}[label=\textbullet]
  \item Планирование roadmap и спринтов, декомпозиция задач
  \item Разработка архитектуры для новой функциональности
  \item Проведение code review
  \item Менторство и обучение сотрудников
  \item Проведенение собеседований разработчиков и аналитиков
  \item Сбор и обработка требований клиентов
  \item Разработка нового функционала и обработка support cases (L3)
  \item Консультация разработчиков прикладных команд по техническим вопросам
\end{itemize}}
\cventry[\cventryvspace]{07.2018~--~07.2019}{}{Autoprint}{Co-founder, CTO (side-project, параллельно с основной работой)}{}{Автоматизация печати документов с использованием IoT, чат-ботов и cloud-технологий.\\
Разработал с нуля ПО для вендингового терминала печати, чат-бота для социальной сети ВКонтакте, сервис приёма платежей для оплаты заказов.\\
Используемые технологии: C\#, .NET Core, Docker, RabbitMQ (MassTransit), EF Core, Serilog, Polly, MediatR, SMNP, IPP, CUPS, Azure IoT Hub, VK API, OpenVPN, OpenWRT, Яндекс.Деньги, xUnit, FluentAssertions, SQL Server, микросервисная архитектура}
\cventry[\cventryvspace]{01.2014~--~07.2017}{}{Acumatica}{Software Developer}{Platform Department}
{Разработка и поддержка \href{https://www.acumatica.com/}{облачной ERP-системы (SaaS) для малого и среднего бизнеса}.
\begin{itemize}[label=\textbullet]
  \item Разработал с нуля такой функционал, как: Dashboards Engine, OLAP / Pivot Tables, OData Endpoint, New UI Frameset, Power BI Integration
  \item Занимался разработкой нового функционала и поддержкой в следующих областях: ORM (SQL Server / MySQL), Analytical Reports Engine (Financial Statements), Plain Reports \& Report Designer, Generic Inquiries, Business Query Language (BQL), Task Scheduler, Authentication \& Authorization, Customization, Excel / PDF Integration, Installation \& Upgrade, Performance Optimizations
\end{itemize}
Основные используемые технологии: C\#, SQL, .NET, ASP.NET MVC, ASP.NET WebAPI, SQL Server, MySQL, Amazon AWS, OWIN, Autofac, Serilog, xUnit, FluentAssertions, Moq, Git, Atlassian stack}
\cventry[\cventryvspace]{06--09.2013}{удалённо}{QB Finance}{Software Developer}{}{Разработка ПО для продвижения в поисковых системах (C\#, Selenium, WinForms, TFS).}
\cventry[\cventryvspace]{06--09.2012}{}{Свой проект}{}{}{Редактор уровней для мобильных игр-платформеров (C\#, XNA 4.0, Farseer Physics Engine, WinForms, Windows Phone, SVN). XML-сериализатор объектного графа c поддержкой кольцевых ссылок и ссылок на файлы для хранения уровней.}
\cventry[\cventryvspace]{07--08.2012}{стажировка}{WebKontrol}{Junior Software Developer}{}{Создание и обновление модулей веб-краулера (C\#, WPF, Entity Framework, Regexp, SVN).}

\section{Образование}
\cventry[\cventryvspace]{2008--2014}{факультет №3 «Системы управления, информатика и электроэнергетика», кафедра 308 «Информационные технологии», специальность «Информационные системы и технологии»}{Московский авиационный институт (МАИ)}{}{}{Поступил в 15 лет (закончил школу экстерном). Средний балл за последние 4 семестра 4.93, общий 4.55.\\Тема диплома: «Разработка системы мониторинга состояния ЛА (Integrated System Health Monitoring) на основе алгоритмов интеллектуального анализа данных (Data Mining)»}
\cventry[\cventryvspace]{2017--2019}{coaching}{Bruce Schoor, SoftO2}{}{Agile Practices, Scrum and Kanban Methods, Product Management~\&~Planning}{}{}
\cventry[\cventryvspace]{2016}{}{Acumatica}{}{F100 Financials: Basic / D100 Distribution: Basic / F350 Analytical Reports / I100 Integration Scenarios / S130 Data Rertrieval and Analysis}{}{}
\cventry[\cventryvspace]{2014}{certificate}{Microsoft}{}{Microsoft Specialist: Programming in C\#}{}{}
\cventry[\cventryvspace]{09--12.2012}{Student of Business Administration (SBA)}{Ассоциация менеджеров России}{}{}{Направление «Инвестиции и инновации».}
\cventry[\cventryvspace]{02--04.2012}{online}{UC Berkeley}{}{}{Курс «Software Engineering for Software as a Service» (Ruby on Rails), балл 2114 из 2126.}

\section{Общестенная деятельность}
\cventry[\cventryvspace]{2014--2015}{}{Молодёжный парламент при Совете депутатов Одинцовского муниципального района Московской области}{Руководитель направления «Информационные технологии, инновации, образование и молодёжное предпринимательство», член Комиссии по формированию Молодёжного парламента}{}{Создание общественной организации, проведение конкурсного отбора её членов, работа над социально-значимыми проектами, участие в процессе законотворчества, проведение переговоров и презентаций, волонтёрская деятельность.}
\cventry[\cventryvspace]{2014, 2016}{}{Благотворительный фонд «Сердце Есть»}{волонтёр}{}{Участие в благотворительных акциях (помощь малоимущим многодетным семьям).}

\section{Конференции и мероприятия}
\cventry[\cventryvspace]{2019}{Москва}{CLRium \#6}{участник}{}{}
\cventry[\cventryvspace]{2019}{Москва}{DotNext 2019 Moscow}{спикер}{}{Roslyn: Мастерство статического анализа}
\cventry[\cventryvspace]{2019}{Москва}{Acumatica Hackathon}{2 место}{}{}
\cventry[\cventryvspace]{2019}{Москва}{Московский Предпринимательский Форум}{участник}{}{}
\cventry[\cventryvspace]{2019}{San Francisco, CA, USA}{Google Cloud Next}{участник}{}{}
\cventry[\cventryvspace]{2019}{New York, NY, USA}{Artificial Intelligence Conference, O'Reilly Media}{участник}{}{}
\cventry[\cventryvspace]{2019}{Houston, TX, USA}{Acumatica Summit}{спикер}{}{}
\cventry[\cventryvspace]{2018}{Москва}{DotNext 2018 Moscow}{участник}{}{}
\cventry[\cventryvspace]{2018}{Москва}{Acumatica Hackathon}{3 место}{}{}
\cventry[\cventryvspace]{2018}{online}{Acumatica Cloud xRP Summit}{спикер}{}{}
\cventry[\cventryvspace]{2018}{Nashville, TN, USA}{Acumatica Summit}{спикер}{}{}
\cventry[\cventryvspace]{2017}{Москва}{Acumatica Hackathon}{1 место}{}{}
\cventry[\cventryvspace]{2017}{San Diego, CA, USA}{Acumatica Summit}{участник, тренинг-ассистент}{}{}
\cventry[\cventryvspace]{2016}{Orlando, FL, USA}{Acumatica Summit}{участник, тренинг-ассистент}{}{}
\cventry[\cventryvspace]{2015}{Москва}{CLRium \#1}{участник}{}{}
\cventry[\cventryvspace]{2014}{Москва}{Живая Профессия}{спикер}{}{Представлял профессию "Software Developer"}

\section{Иностранные языки}
\cvitem{\textbf{Английский}}{Свободное владение}

\section{Дополнительные сведения}
\cvitem[\cventryvspace]{Возраст}{26 лет}
\cvitem[\cventryvspace]{Личные качества}{Быстро обучаюсь, легко работаю в команде, спокойно беру на себя ответственность. Умею самостоятельно организовывать свою работу и работу других. Решаю конфликты, а не создаю их. Не употребляю алкоголь, не курю, вегетарианец более 10 лет.}
\cvitem[\cventryvspace]{Хобби}{Прикладная психология, танцы, спорт, гитара, путешествия, здоровый образ жизни.}
%\clearpage
\end{document}