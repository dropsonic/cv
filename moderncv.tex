\documentclass[11pt,a4paper,roman,russian]{moderncv} % possible options include font size ('10pt', '11pt' and '12pt'), paper size ('a4paper', 'letterpaper', 'a5paper', 'legalpaper', 'executivepaper' and 'landscape') and font family ('sans' and 'roman')
\moderncvstyle{modernclassic} % style options are 'casual' (default), 'classic', 'oldstyle' and 'banking'
\moderncvcolor{blue}                               % color options 'blue' (default), 'orange', 'green', 'red', 'purple', 'grey' and 'black'
%\nopagenumbers{}                                  % uncomment to suppress automatic page numbering for CVs longer than one page
\usemoderncventry

\usepackage[scale=0.8,a4paper]{geometry}
\usepackage{xltxtra}
\usepackage{xunicode}
\setmainfont[Mapping=tex-text]{Times New Roman}
\setsansfont[Mapping=tex-text]{Calibri}
\newfontfamily{\cyrillicfonttt}{Courier New}
\usepackage{polyglossia}
\setmainlanguage{english}
%\setotherlanguage[variant=us]{russian}

%\newcommand\cventryitem{\item[\textbullet]\hspace*{1em}}
\usepackage{enumitem}
\setlist{nolistsep} % убрать лишний интервал между элементами списка
\setlist[itemize,1]{labelindent=6pt, leftmargin=*}
\newcommand{\cventryitem}{\item[\textbullet]}

\newlength{\cventryvspace}
\setlength{\cventryvspace}{0.75em}

%----------------------------------------------------------------------------------
%            personal data
%----------------------------------------------------------------------------------
\firstname{Владимир}
\familyname{Панченко}
\title{Software Developer}
\vacancy{C\# разработчик (Unity3D)}
%\address{г. Москва}
\mobile{+7 (926) 410-55-70}
\email{volodymyrthegreat@live.com}
\skype{volodymyrthegreat}
\github{dropsonic}

\photo[100pt][0pt]{picture}
\bannerquote{"Any fool can write code that a computer can understand. Good programmers write code that humans can understand."~---~Martin~Fowler}
%

\begin{document}
%-----       resume       ---------------------------------------------------------
\makecvtitle

\section{Ключевые навыки}
\cvitem{}{C\# 5.0 (.NET 4.5.1)  / ASP.NET / WPF / XNA / Entity Framework / Windows Phone / \LaTeX / WCF / MS Unit Testing Framework / XML / Ruby / Ruby on Rails / Delphi}
\cvitem{}{Git, TFS, SVN, VS2010/2012/2013, ReSharper}
\cvitem{}{Хороший уровень владения ООП, шаблонами проектирования, рефакторингом, UML. Знание Agile-методологий, TDD.}
\section{Опыт работы}
\cventry[\cventryvspace]{01.2014~--~н.в.}{}{Acumatica}{}{}{Разработка и поддержка cloud application framework и ядра облачной ERP-системы (C\#, ASP.NET, T-SQL, TFS).}
\cventry[\cventryvspace]{06--09.2013}{удалённо}{QB Finance}{}{}{Разработка ПО для продвижения в поисковых системах (C\#, Selenium, WinForms, TFS).}
\cventry[\cventryvspace]{06--09.2012}{}{Свой проект}{}{}{Редактор уровней (C\#, XNA 4.0, Farseer Physics Engine, WinForms, Windows Phone, SVN). Свой XML-сериализатор объектного графа c поддержкой кольцевых ссылок и ссылок на файлы.}
\cventry[\cventryvspace]{07--08.2012}{стажировка}{WebKontrol}{}{}{Создание и обновление модулей веб-краулера (C\#, WPF, Entity Framework, Regexp).}
\cventry[\cventryvspace]{06--07.2012}{практика}{НИИ точных приборов (ОАО «НИИ ТП»)}{}{}{Разработка геоинформационных систем (C\#, SVN).}
\cventry[\cventryvspace]{06--07.2011}{практика}{НПО им. Лавочкина}{}{}{Моделирование спутниковых систем.}
\cventry[\cventryvspace]{06--07.2010}{практика}{Центральный научно-исследовательский институт автоматики и гидраливики (ОАО «ЦНИИАГ»)}{}{}{Портирование и создание средств разработки управляющих программ ракетных комплексов (Delphi).}
\cventry[\cventryvspace]{06--12.2010}{техник каф. 308}{Московский авиационный институт (МАИ)}{}{}{Разработка и сопровождение систем обработки информации (Delphi).}

\section{Образование}
\cventry[\cventryvspace]{2008--2014}{факультет №3 «Системы управления, информатика и электроэнергетика», кафедра 308 «Информационные технологии», специальность «Информационные системы и технологии»}{Московский авиационный институт (МАИ)}{}{}{Поступил в 15 лет (закончил школу экстерном). Средний балл за последние 4 семестра 4.93, общий 4.55.}
\cventry[\cventryvspace]{09--12.2012}{Student of Business Administration (SBA)}{Ассоциация менеджеров России}{}{}{Направление «Инвестиции и инновации».}
\cventry[\cventryvspace]{02--04.2012}{online}{UC Berkeley}{}{}{Курс «Software Engineering for Software as a Service» (Ruby on Rails), балл 2114 из 2126.}

\section{Иностранные языки}
\cvitem{\textbf{Английский}}{Разговорный, свободное чтение любой технической литературы.}

\section{Дополнительные сведения}
\cvitem[\cventryvspace]{Возраст}{20 лет}
\cvitem[\cventryvspace]{Личные качества}{Коммуникабелен, бесконфликтен, готов (и хочу!) брать на себя ответственность. Высокий уровень грамотности.}
\cvitem[\cventryvspace]{Хобби}{Прикладная психология, музыка (электрогитара), спорт.}
\clearpage

%-----       letter       ---------------------------------------------------------
% recipient data
\recipient{Ивану Ивановичу Иванову}{ООО «Рога и копыта»\\Москва, ул. Ленина, 27\\123456}
\date{1 марта 2014}
\opening{Уважаемый Иван Иванович,}
\closing{С наилучшими пожеланиями,}
\makelettertitle
\makeletterclosing
\end{document}
